\documentclass[a4paper, 11pt]{article}
	
	\usepackage[utf8]{inputenc}
	\usepackage[T1]{fontenc}
	\usepackage[french]{babel}
	\usepackage{amsmath}

	\title{Mémo Métrologie}
	\author{L1SPI\\
        Largement inspiré du cours de métrologie S1\&2\\
        par B. Lihoreau \& C. Ayrault}
	\date{Avril 2012}

\begin{document}
	\maketitle
	
\section{Définitions}
\begin{description}
    \item[Erreur systématique] Sur/sous-estimation systématique de la valeur vraie du mesurande.
        Liée aux imperfection du dispositif (appareils, etc...) ou procédé de mesure.
    \item[Erreur Aléatoire] Sur/sous-estimation aléatoire de la valeur vraie du mesurande.
        Le {\it "décalage"} par rapport à la valeur vraie n'est pas nécessairement le même d'un mesure sur l'autre.
    \item[Justesse] Qualité d'un appareil vis-à-vis des erreurs systématiques
    \item[Fidelité] Qualité d'un appareil vis-à-vis des erreurs aléatoires
    \item[Précision] Qualité d'un appareil à la fois juste et fidèle
    \item[Mesure Directe] Comparaison avec un étalon
    \item[Mesure Indirecte] Détermination de la valeur du mesurande par mesurage sucessif d'autres grandeurs et calcul 
    \item[Erreur Absolue] $$\Delta G = \left| G_{vraie} - G_{mes} \right|$$
    \item[Intervale de confiance] (ici) $$IC = 2\Delta G$$
    \item[Erreur Relative] $$E.R. = \frac{G_{vraie} - G_{mes}}{G_{vraie}}$$
    \item[Incertitude relative] $$I.R. = \frac{\Delta G}{G_{mes}}$$
\end{description}

\section{Sources d'erreurs}
\begin{description}
    \item[Lecture] Compter une demi-division sur un appareil analogique et le dernier digit sur du matériel numérique
    \item[Limite de précision de l'appareil] Voir la classe de l'appareil et plus généréralement, sa documentation complète
    \item[Définition] Qu'est ce que la "position" d'un capteur ? (celui ci étant plus ou moins instrusif), etc...
\end{description}

Dans tous les cas :
\begin{itemize}
    \item Utiliser le meilleur calibre/dynamique (au plus proche du signal)
    \item On peut souvent négliger une 2 des sources d'erreurs, faibles devant la troisième
\end{itemize}


\section{Réduction des erreurs (mesures directes)}
\subsection{Systématique}
\begin{itemize}
    \item Analyse du processus de mesure
    \item Etalonnage des appareils
\end{itemize}

\subsection{Aléatoire}
\subsubsection{Outils Statistiques}

Répétition des mesures puis moyenne :

$$\bar x = \frac{1}{N}\sum_{i=1}^{N}x_i$$

La dispersion par rapport à $\bar x$ est donnée par l'écart type\footnote{On peut reconstruire la formule à partir de la phrase {\it "La racine carrée de la moyenne des écarts à la moyenne au carré"}} :

$$\sigma = \sqrt{\frac{1}{N}\sum_{i=1}^{N}(x_i - \bar x)^2}$$

\section{Incertitudes avec une lois de comportement (mesures indirectes)}

On part du fait que :
\begin{itemize}
    \item on a $G = f(x_1, x_2,\ldots,x_n)$
    \item on connait les incertitudes absolues $\Delta x_i$
    \item on cherche $\Delta G$
\end{itemize}

\subsection{Loi de propagation des erreurs maximales}

Si on part du fait que $\Delta x_i << x_i$ alors on majore $\Delta G$ par : 
$$\Delta G = \sum_{i=1}^n \left|\frac{\partial G}{\partial x_i}\right| \Delta x_i$$

\subsection{Calcul incertitudes-types composées}

$$\Delta G = \sqrt{\sum_{i=1}^n\left(\frac{\partial G}{\partial x_i}\right)^2\left(\Delta x_i\right)^2}$$


\subsection{Règles pour les cas simples}

%\begin{tabular}{|c|c|c|}\hline
%Loi & $y = x_1\pm x_2$ & $y=x_1^mx_2^n$\\\hline
%Erreurs maximales & $\Delta y = \Delta x_1 + \Delta x_2$ & $\frac{\Delta y}{y} = |m|\frac{\Delta x_1}{x_1} + |n|\frac{\Delta x_2}{x_2}$\\\hline
%Incertitudes-types composées & $\Delta y = \sqrt{\left((\Delta x_1\right)^2 + \left(\Delta x_2\right)^2}$ & $\frac{\Delta y}{y} = \sqrt{m^2\left(\frac{\Delta x_1}{x_1}\right)^2+n^2\left(\frac{\Delta x_2}{x_2}\right)^2}$\\\hline
%\end{tabular}

\begin{tabular}{c||c|c}
Loi de comportement & Erreurs maximales & Incertitudes-types Composées \\\hline\hline
$y = x_1\pm x_2$ & $\Delta y = \Delta x_1 + \Delta x_2$ & $\Delta y = \sqrt{\left((\Delta x_1\right)^2 + \left(\Delta x_2\right)^2}$ \\\hline
$y=x_1^mx_2^n$ & $\frac{\Delta y}{y} = |m|\frac{\Delta x_1}{x_1} + |n|\frac{\Delta x_2}{x_2}$ & $\frac{\Delta y}{y} = \sqrt{m^2\left(\frac{\Delta x_1}{x_1}\right)^2+n^2\left(\frac{\Delta x_2}{x_2}\right)^2}$
\end{tabular}

\end{document}
