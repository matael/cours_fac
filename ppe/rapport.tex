\documentclass[a4paper, 11pt]{report}
    \usepackage[utf8]{inputenc}
    \usepackage[T1]{fontenc}
    \usepackage[french]{babel}

    \title{{\Huge Rapport}\\Projet Personnel de l'Etudiant\\
        Salles de Spectacle}
    \author{Mathieu Gaborit — Jean-Suliac Defontaine\\Professeur : Christophe Ayrault}
    \date{Université du Maine | Mai 2012} 

\begin{document}
    \fontfamily{phv}
    \selectfont

    \maketitle
    
    % table of contents
    \tableofcontents
    \newpage


% Section : Introduction
\section*{Introduction}

Au cours de cette première année, il nous a été demandé de préparer une présentation de notre projet professionel (ou du moins d'un projet possible).
En définitive, il s'agissait d'une recherche concernant un métier et d'une prise de contact avec un professionnel.

Le Projet Personnel de L'Etudiant (PPE) présenté ici est principalement une recherche autour de la conception des salles de spectacle.

Le choix a été fait de croiser plusieurs sources d'informations, de se baser sur nos connaissances, l'Internet et une interview formelle (d'autre personnes ont été contactées avant ou pendant).

L'objectif était, plutôt que de travailler que sur un point de vue, d'en confronter plusieurs.

L'un de nous a déjà évolué dans le monde du spectacle côté technique, l'autre, musicien, connait aussi la scène.
Nous avions chacun une affinité plus ou moins marquée avec cet univers et nous avons cherché à connaître les points de vue de personnes impliquées dans la création de salles.

\section{Au commencement}

Avant toute chose, il a été convenu de recenser les informations déjà en notre possession et de définir des objectifs.

Les informations {\em fiables} à proprement parler n'étaient pas forcément très nombreuses mais certains de nos a priori se sont vite confirmés.

\subsection{Avant : le point de vue du technicien}

Avant toute recherche, nous nous sommes concertés quand à l'implication des techniciens (futurs utilisateurs) dans la conception de la salle.

Après quelques prestations, il apparait que les salles sont plus souvent faite pour être {\em belles} que pour être {\em pratiques}.
Ce postulat est tiré directement de nos expériences ainsi que de quelques discussions antérieures avec des professionnels du monde du spectacle.

Afin de vérifier cette proposition, une interview de Lilian Duffrechoux a été réalisée.

\subsection{Avant : le point de vue de l'acousticien}

Vis à vis de l'acousticien, nous pensions qu'il était {\em systématiquement} consulté avant toute construction.
Consulté ne signifie pas forcément écouté...

\subsection{Avant : le point de vue de l'architecte}

Maître de la conception d'un batiment, le point de vue de l'architecte semble primordial.
Avant d'aller plus loin, nous pensions que les architectes concevant les salles de spectacle étaient spécialisés, informés au moins.

Il n'était pas absurde de penser qu'une formation à l'acoustique des salles leur était dispensée.

\section{Résultats}

AJOUTER INTERVIEW

\section{Discussion}

Les recherches menées et l'interview en elle même mettent en évidence des failles dans notre conception de ce groupe de métiers.

M. Duffrechoux nous a clairement signifié que les techniciens n'étaient que rarement consultés durant la conception du salle de spectacle.
L'exemple du Grand Atelier (Espace de Vie Etudiante de l'Université du Maine) a été pris notament pour montrer que  l'ITEMM n'avait pas été consulté.

Un précédent échange avec un membre du département acoustique de la faculté des sciences avait déjà mis en évidence le non respect des recommendations fournies.

Enfin, lors d'interventions au Grand Atelier, des étudiants de l'ENSIM (faisant alors des tests) avaient confirmé la piètre qualité de la salle en dépit des personnes compétentes présentes sur place.

\medskip

Un certain nombre de personnes appartenant au monde du spectacle nous ont eux aussi confirmé qu'ils n'avaient été consultés qu'a posteriori, généralement lors de la phase d'équipement de la salle.

La formation des architectes comporte une introduction sommaire à l'acoustique, celle ci pouvant être approfondie au cours d'une spécialisation.
A noter que cette spécialisation n'est pas obligatoire pour la construction de salles de spectacle.


Selon M. Duffrechoux, un bureau d'architecture ne comprend pas forcément (et même rarement) d'acousticien et fait donc appel à un bureau d'étude spécialisé le cas échéant (ou pas).

\section{Enfin}

A la lumière de ces informations, il nous a fallu revoir notre conception des liens entre ces métiers.

On constate que, la plupart du temps, la seule chose qui les lie vraiment est la salle en elle-même.

A titre personnel, ce qui m'a semblé le plus abérant est la non-consultation courante d'un bureau d'étude acoustique au cours de la conception.
Bien sûr, et on le remarque particulièrement lors de l'utilisation de la salle, il en résulte une acoustique parfois médiocre et mal optimisée.
Un second effet est parfois plus gênant : le monde du spectacle court en permanence contre la montre, une mauvaise acoustique de base occasionne des pertes de temps parfois monstrueuses.

Un aspect qui n'a été que peu développé ici est le point de vue du mandataire lui-même.
En effet le {\emph "client"} n'a pas (du moins pas toujours) un budget extensible, et l'on doit donc prendre en compte cette contrainte : on préfèrera souvent quelque chose de beau à voir que de beau à entendre.

Reste à examiner un problème récurrent dans les cas plus modestes : les demandeurs souhaitent parfois (souvent ?) réutiliser un batiment existant pour base.
Cela évite en effet une bonne partie d'un gros-oeuvre souvent couteux.

Reste quand même qu'avec toute la bonne volonté du monde, les miracles n'existent pas : on ne transforme pas une ancienne conserverie de poisson en salle de spectacle (Saint Gilles Croix de Vie, Vendée) ou bien une gare en théatre (Fontenay le Comte), etc...


Ces obstacles purement pécuniers sont autant de freins à la réalisation de salles à la fois belles et agréables (tant à l'oreille que pour y travailler).

\end{document}
