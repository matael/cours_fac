\documentclass[a4paper]{report}
% generated by Docutils <http://docutils.sourceforge.net/>
\usepackage{fixltx2e} % LaTeX patches, \textsubscript
\usepackage{cmap} % fix search and cut-and-paste in Acrobat
\usepackage{ifthen}
\usepackage[T1]{fontenc}
\usepackage[utf8]{inputenc}
\usepackage{color}
\usepackage{longtable,ltcaption,array}
\setlength{\extrarowheight}{2pt}
\newlength{\DUtablewidth} % internal use in tables

%%% Custom LaTeX preamble
% PDF Standard Fonts
\usepackage{mathptmx} % Times
\usepackage[scaled=.90]{helvet}
\usepackage{courier}

%%% User specified packages and stylesheets

%%% Fallback definitions for Docutils-specific commands

% admonition (specially marked topic)
\providecommand{\DUadmonition}[2][class-arg]{%
  % try \DUadmonition#1{#2}:
  \ifcsname DUadmonition#1\endcsname%
    \csname DUadmonition#1\endcsname{#2}%
  \else
    \begin{center}
      \fbox{\parbox{0.9\textwidth}{#2}}
    \end{center}
  \fi
}

% title for topics, admonitions and sidebar
\providecommand*{\DUtitle}[2][class-arg]{%
  % call \DUtitle#1{#2} if it exists:
  \ifcsname DUtitle#1\endcsname%
    \csname DUtitle#1\endcsname{#2}%
  \else
    \smallskip\noindent\textbf{#2}\smallskip%
  \fi
}

% hyperlinks:
\ifthenelse{\isundefined{\hypersetup}}{
  \usepackage[colorlinks=true,linkcolor=blue,urlcolor=blue]{hyperref}
  \urlstyle{same} % normal text font (alternatives: tt, rm, sf)
}{}
\hypersetup{
  pdftitle={Audiologie},
}

%%% Title Data
\title{\phantomsection%
  Audiologie%
  \label{audiologie}}
\author{Erwan Marchand \and Mathieu Gaborit}
\date{Mai 2012}

%%% Table of contents

\setcounter{tocdepth}{4}
\renewcommand{\contentsname}{Demandez le programme}

%%% Body
\begin{document}
\maketitle

\tableofcontents
\newpage

L'objectif de ce TP est d'introduire quelques tests simples permettant la mesure de l'acuité auditive et le diagnostic de pathologies auditives simples.
Nous en profiterons pour évaluer l'efficacité de protections auditives bon marché (un TP sponsorisé par E.A.R, fabricant de bouchons d'oreille).


%___________________________________________________________________________

\section*{\phantomsection%
  Test diagnostique au diapason%
  \addcontentsline{toc}{section}{Test diagnostique au diapason}%
  \label{test-diagnostique-au-diapason}%
}

Ces deux tests au diapason permettent un diagnostic sommaire des pertes auditives.
Réalisables en quelques minutes, il mettent en évidence les défauts unilatéraux de l'audition, cernant ainsi l'oreille défectueuse.


%___________________________________________________________________________

\subsection*{\phantomsection%
  Test de Rinné%
  \addcontentsline{toc}{subsection}{Test de Rinné}%
  \label{test-de-rinne}%
}

Le test de Rinné est réalisé sur chacun de nous deux.

Il consiste à frapper un diapason, puis à se l'appliquer sur le front.
Le patient indique ensuite s'il perçoit le son à gauche, à droite, ou au milieu.

On entend mieux du côté défectueux si la perte vient d'une mauvaise conduction aérienne, et du bon côté si la surdité est liée à une perte neuro-sensorielle.

Un patient percevant le son au centre dispose d'une audition normale.


%___________________________________________________________________________

\subsubsection*{\phantomsection%
  Note : Résultats%
  \addcontentsline{toc}{subsubsection}{Note : Résultats}%
  \label{note-resultats}%
}

\begin{center}
\begin{tabular}{|c|c|}\hline
Résultat Erwan & milieu\\\hline
Résultat Mathieu & milieu\\\hline
\end{tabular}
\end{center}

Moralité : nous sommes encore vaillants.


%___________________________________________________________________________

\subsection*{\phantomsection%
  Test de Weber%
  \addcontentsline{toc}{subsection}{Test de Weber}%
  \label{test-de-weber}%
}

Le test de Weber se réalise en plaçant un diapason excité contre une mastoïde.
Lorsque l'on n'entend plus le son, on place alors le diapason près de l'oreille, si le patient entend, il a une audition normale ou souffre de pertes neuro-sensorielles.
Dans le cas contraire, la conduction aérienne est mauvaise.


%___________________________________________________________________________

\subsubsection*{\phantomsection%
  Note: Résultats%
  \addcontentsline{toc}{subsubsection}{Note: Résultats}%
  \label{id1}%
}


%___________________________________________________________________________

\paragraph*{\phantomsection%
  Sans bouchon%
  \addcontentsline{toc}{paragraph}{Sans bouchon}%
  \label{sans-bouchon}%
}

\begin{center}
\begin{tabular}{|c|c|}\hline
Erwan Gauche & OK\\\hline
Erwan Droite & OK\\\hline
Mathieu Gauche & OK\\\hline
Mathieu Droite & OK\\\hline
\end{tabular}
\end{center}

On remarque que nos oreilles sont plutôt bien conservées.


%___________________________________________________________________________

\paragraph*{\phantomsection%
  Avec bouchon%
  \addcontentsline{toc}{paragraph}{Avec bouchon}%
  \label{avec-bouchon}%
}

Comme nos deux ``expérimentateurs'' ont des oreilles potables, nous simulons une perte aérienne avec un bouchon.

Erwan (qui a fait le test) remarque clairement la différence.

Nous en concluons que ce test fonctionne, la perte par voie aérienne est détectable.

Pour ce qui est d'une éventuelle perte neuro-esnsorielle, elle est difficilement caractérisable par ce genre de tests (Rinné et Weber), tout au plus détectable.


%___________________________________________________________________________

\section*{\phantomsection%
  Audiométrie%
  \addcontentsline{toc}{section}{Audiométrie}%
  \label{audiometrie}%
}

Afin de jouer chacun notre tour, nous avons réalisé l'audiogramme de l'oreille droite d'Erwan et celui de l'oreille gauche de Mathieu.

Pour évaluer l'efficacité des protections, c'est l'oreille droite d'Elliot que nous avons torturé sans, puis avec un bouchon.
Son oreille gauche a aussi été testée pour pouvoir tracer un audiogramme complet.

Enfin, Thomas nous a prêté sa mastoïde droite pour réaliser le test en conduction osseuse.

Nous pensons que nos mesures ont été fortement influencées par la qualité vestute de l'audiomètre.
Loin de nous l'envie de dénigrer le sus-cité appareil, mais force est de constater qu'un outil dont le casque est cassé et dont un des deux écouteurs ne fonctionne pas (le gauche), ne nous inspire pas confiance.
Ces mesures sont donc à juger avec réserve.


%___________________________________________________________________________

\subsection*{\phantomsection%
  Audiogrammes%
  \addcontentsline{toc}{subsection}{Audiogrammes}%
  \label{audiogrammes}%
}

L'annexe 1 montre une superposition de 4 audiogrammes :
%
\begin{itemize}

\item Erwan : Oreille Droite

\item Mathieu : Oreille Gauche

\item Elliot : Oreilles Droite et Gauche

\end{itemize}

On note qu'Elliot perçoit des sons largement moins intenses que Mathieu et Erwan : en effet, sa courbe est loin ``en dessous'' des deux autres.
Il a probablement une meilleure audition que les deux autres.

Erwan semble être celui dont les pertes auditives sont les plus importantes.

On remarque que ces courbes ressemblent aux courbes d'isosonie/isophonie.


%___________________________________________________________________________

\subsection*{\phantomsection%
  Pertes%
  \addcontentsline{toc}{subsection}{Pertes}%
  \label{pertes}%
}

\begin{center}
\begin{tabular}{|c|c|c|c|c|}
\hline
  Fréquence     & OD Elliot &  OG Elliot & OD Erwan & OG Mathieu \\
\hline
    500         &     0     &     0      &    25    &     10     \\
   1000         &     0     &     0      &     0    &      5     \\
   2000         &     0     &     0      &    15    &     10     \\
   3000         &     0     &     0      &    20    &     20     \\
\hline
\hline
 Pertes Audio.  &     0     &     0      &    15    &   11,25    \\
\hline
\hline
 Pertes en \%    &     0     &     0      &   22,5   &     17     \\
\hline
\end{tabular}
\end{center}

Ces valeurs confirment les hypothèses formulées précédement par rapport à l'acuité auditive des sujets.

La perte binaurale est calculée sur les mesures pour Elliot :

\begin{center}
\begin{tabular}{|c|c|c|}
\hline
   Fréquence    &    Oreille Droite   &  Oreille Gauche \\
\hline
     500        &        0            &        0        \\
    1000        &        0            &        0        \\
    2000        &        0            &        0        \\
    3000        &        0            &        0        \\
\hline
\hline
 Pertes Audio.  &        0            &        0        \\
\hline
\hline
 Pertes en \%    &        0            &        0        \\
\hline
\end{tabular}
\end{center}

Perte binaurale : (0+0)/6 = 0

Ce sujet de test est un peu frustrant...


%___________________________________________________________________________

\subsection*{\phantomsection%
  Conduction Osseuse%
  \addcontentsline{toc}{subsection}{Conduction Osseuse}%
  \label{conduction-osseuse}%
}

Thomas s'est prêté à l'expérience (ou plutot, nous a prêté sa mastoïde...) pour la caractérisation de la conduction osseuse.

Nous en tirons les mesures présentées en annexe 2.

D'après la courbe, la conduction osseuse semble particulièrement efficiente pour les fréquences moyennement élevées.

Le sujet (alias Thomas) nous a clairement fait comprendre qu'il ne percevait rien au dessus de 4000Hz.


%___________________________________________________________________________

\subsection*{\phantomsection%
  Oreille saine, Oreille bouchée%
  \addcontentsline{toc}{subsection}{Oreille saine, Oreille bouchée}%
  \label{oreille-saine-oreille-bouchee}%
}

Elliot retourne sur le siège pour se soumettre aux même mesures qu'auparavant en se bouchant l'oreille droite au moyen d'un bouchon d'oreille.

On obtient la courbe présentée en bleu sur l'annexe 3, celle en noir étant la courbe de l'oreille saine.

La courbe bleue est très nettement ``au-dessus'' de la courbe noire : avec un bouchon, les seuils d'audition sont largement plus élevés.

La courbe rouge montre la différence entre les seuils avec le bouchon et sans le bouchon : plus elle est élevée, plus le bouchon est efficace à la fréquence donnée.

On remarque que cette protection est beaucoup plus efficace en hautes fréquences.
Une hypothèse pourrait être qu'en basses fréquences, le bouchon est plus petit que la longueur d'onde et qu'il n'est pas assez ``gênant''.

On cherche ensuite à déterminer la perte de chaque oreille et la perte binaurale liée à l'utilisation du bouchon.

\begin{center}
\begin{tabular}{|c|c|c|}
\hline
   Fréquence    & Oreille Droite (bouchon) &  Oreille Gauche \\
\hline
     500        &           25             &        0        \\
    1000        &           15             &        0        \\
    2000        &           25             &        0        \\
    3000        &           25             &        0        \\
\hline
\hline
 Pertes Audio.  &          22.5            &        0        \\
\hline
\hline
 Pertes en \%    &          33.75           &        0        \\
\hline
\end{tabular}
\end{center}

Perte binaurale : (0*5+33.75)/6 = 5.625\%

Le bouchon occasione une perte non négligeable.
S'il n'est pas une protection parfaite, il reste efficace et un bon début dans la protection bon-marché.


%___________________________________________________________________________

\section*{\phantomsection%
  Conclusion%
  \addcontentsline{toc}{section}{Conclusion}%
  \label{conclusion}%
}

En conclusion, ce TP nous a permit de comprendre les tests auditifs pratiqués pour le dépistage de pathologies auditives (nobostant un audiomêtre pitoyablement mauvais).
Nous avons en outre pu constater que les protections auditives (même bon marché) sont relativement efficaces.

Un test intéressant à réaliser serait de mesurer l'efficience de bouchons sur-mesure et de la comparer à celle de bouchons standard.

Ce TP permet aussi de faire le lien avec les courbes vues en cours/TD à propos de la sonie et de l'intensité subjective des sons.

\end{document}
