\documentclass[a4paper, 11pt]{article}

    \usepackage[utf8]{inputenc}
    \usepackage[T1]{fontenc}
    \usepackage[french]{babel}
    \usepackage{amsmath}
    
    \title{Electronique — Mémo}
    \author{License SPI — S2}
    \date{Mars 2012}

\begin{document}
    \maketitle

    \section{Loi des mailles}

    $$\sum U = 0$$

    Dans une maille, la somme des tensions est nulle.

    \section{Loi des noeuds}

    $$\sum I_e = \sum I_s$$

    Pour un noeud, la somme des courants entrants est égale à la somme des courants sortants.

    \section{Résistances}

        \subsection{Loi d'Ohm}

        $$U=RI$$

        \subsection{Résistance équivalente}

        \subsubsection{Série}

        $$R_{eq} = R_1 + R_2 + \cdots + R_n = \sum_{i=0}^n R_i$$

        \subsubsection{Parallèle}

        $$\frac{1}{R_{eq}} = \frac{1}{R_1} + \frac{1}{R_2} + \cdots + \frac{1}{R_n} = \sum_{i=0}^n \frac{1}{R_i}$$

        \paragraph{Cas pour 2 résistances en parallèle}

        $$R_{eq} = \frac{R_1R_2}{R_1+R_2}$$

    \section{Capacités}

        \subsection{Tension}

        $$CU_C=q$$

        \subsection{Intensité}

        Dans le sens {\bf direct} :

        \begin{eqnarray*}
            i & = & \frac{dq}{dt} \\
            q & = & CU_C \\
            i & = & C\frac{dU_C}{dt}
        \end{eqnarray*}

        \subsection{Impédance}

        $$Z_C = \frac{1}{j\omega C}$$

        \subsection{Capacité équivalente}
        
        \subsubsection{Série}

        $$\frac{1}{C_{eq}} = \frac{1}{C_1} + \frac{1}{C_2} + \cdots + \frac{1}{C_n} = \sum_{i=0}^n \frac{1}{C_i}$$

        \subsubsection{Parallèle}

        $$C_{eq} = C_1 + C_2 + \cdots + C_n = \sum_{i=0}^n C_i$$

    \section{Inductances}
    
        \subsection{Tension}

        $$V_L = L\frac{di}{dt}$$

        \subsection{Impédance}

        $$Z_L = jL\omega$$

    \section{Impédances}

        \subsection{Définition}

        Soient $I$, $U$ et $Z$ des nombres complexes modélisant les grandeurs sinusoïdales réelles $i$ et $u$ ainsi que l'impédance.

        \begin{eqnarray*}
            I(t) & = & I_{MAX}e^{j\omega t}\\
            U(t) & = & U_{MAX}e^{j(\omega t+\phi )} \\
            i(t) & = & Re(I) = I_{MAX}\cos(\omega t) \\
            u(t) & = & Re(u) = U_{MAX}\cos(\omega t + \phi)\\
            U & = & ZI \\
            \left| Z\right| & = & \frac{U_{MAX}}{I_{MAX}} \\
            U_{EFF} & = & \frac{U_{MAX}}{\sqrt{2}}\\
            arg(Z) & = & \phi
        \end{eqnarray*}

\end{document}
