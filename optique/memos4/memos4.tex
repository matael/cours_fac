\documentclass[a4paper, 11pt]{article}

	\usepackage[utf8]{inputenc}
	\usepackage[T1]{fontenc}
	\usepackage[french]{babel}
	\usepackage{amsmath}

    \title{Memo -- Optique S4}
    \author{Mathieu Gaborit}

    \renewcommand{\bar}{\overline}


\begin{document}
	\maketitle

\section{Définitions}

\begin{description}
    \item[Foyer image] Noté $F'$, endroit où se forme l'image d'un objet à l'infini
    \item[Foyer objet] Noté $F$, si l'on y place un objet, son image se forme à l'infini
    \item[Distance focale] Notée $f'$, distance (algébrique) $\bar{OF'}$
    \item[Vergence] Notée $C$ (en \textit{dioptries} $\delta$ ou $m^{-1}$) et vaut $\frac{1}{f'}$
\end{description}

\section{Formules générales}

\subsection{Formule de conjugaison}

$$\frac{1}{\bar{OA'}} - \frac{1}{\bar{OA}} = \frac{1}{\bar{OF'}} = \frac{1}{f'}$$

$$\bar{FA}\cdot\bar{F'A'} = -f'^2$$

\subsection{Formule de grandissement}

$$\gamma = \frac{\bar{A'B'}}{\bar{AB}} = \frac{\bar{OA'}}{\bar{OA}}$$

Si $\gamma < 0$ alors l'image est à l'envers.

\section{Lentilles}

\begin{description}
    \item[Convergente] $f' > 0$ ; les rayons venant de l'infini convergent vers le point focal image (situé
        \textit{derrière} la lentille
    \item[Divergente] $f' < 0$ ; les rayons venant l'infini divergent depuis le point focal image (situé \textit{devant}
        la lentille
\end{description}

\end{document}
