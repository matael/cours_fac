\section{Organisation}

\subsection{L'équipe pédagogique}

Tout au long de l'UEL, nous étions 4 étudiants à intervenir en même temps au lycée Yourcenar. Parmi les trois autres, il
y avait un camarade de promotion (Baptiste) et deux étudiants en langues (Anglais).

Nous nous sommes insérés dans une équipe pédagogique constituée de 6 professeurs (Anglais, Français, Maths,
Histoire-Géographie, SES --- Sciences Economiques et Sociales, et SVT --- Sciences de la Vie et de la Terre). Nous
avons été dans l'ensemble bien accueillis et avons vite pris nos marques.

\subsection{Les trois temps}

\subsubsection{Travail personnel et soutien}

Dans ce premier temps (qui a duré 6 semaines), les élèves devaient préparer un mini TPE (pour s'entrainer pour la classe
de première). Nous les avons aidé à problématiser leur sujet au cours de la première séance et, pendant les 5 suivantes,
nous avons assuré du soutien parallèlement à leur travail personnel.

Avec Baptiste, nous assurions un soutien en maths, sur deux séances d'une heure (avec deux groupes différents). Les
groupes pour chacune des séances changeaient presque chaque semaine, aussi aucun réel suivi n'a été mis en place. Nous
accueillions à chaque heure une dixaine d'élèves.

Sachant que les élèves venait souvent pour poser des questions sur le cours de la semaine, nous avons pris le parti
qu'un d'entre nous se chargerait de faire cours et exercices au tableau pendant la première heure sur le sujet intéressant
le plus grand nombre tandis que l'autre passerait aider dans les rangs où répondrait à d'autres questions. Pour la
deuxième heure nous échangerions les rôles.

Cette façon de faire semble avoir bien fonctionné dans l'ensemble.

\subsubsection{Soutenance des mini-TPE}

L'équipe pédagogique avait décidé de faire passer des soutenances pour le mini-TPE afin de former les élèves à cet
exercice parfois difficile.

Il a été décidé que nous nous répartirions dans les différents jurys.

C'est ainsi que pendant 3 semaines, nous avons pu assister aux soutenances en tant qu'examinateurs et participer à la
notation.

Malheureusement pour nous, étudiants en physique, aucun groupe n'a traité de sujet nous concernant particulièrement
(beaucoup de groupes traitaient de sciences humaines et économiques, certains de mathématiques).

\subsubsection{Les deux dernières semaines}

Au cours des deux dernières semaines, les élèves devaient produire quelque chose (dossier, siteweb, BD, video, etc...)
pour conseiller les futurs secondes dans leur vie au lycée.

J'ai passé une des deux séances au SUIO-IP pour la réunion autour du "portefeuille de compétences" et la seconde au
lycée où nous avons fait de la surveillance. C'est très probablement la séance la moins intéressante de l'UEL.
