\section{Introduction}

Au cours de ce dernier semestre de L3, il nous a été demandé de choisir une UEL.

L'accompagnement d'élèves en difficulté en lycée m'a semblé un bon choix pour plusieurs raisons :

\begin{itemize}
    \item l'aide scolaire n'était pas une première pour moi et avait toujours été une bonne expérience ;
    \item j'apprécie de transmettre des connaissances ;
    \item travailler avec une équipe pédagogique me semblait être une expérience enrichissante.
\end{itemize}

Mon choix d'établissement s'est porté sur le lycée Marguerite Yourcenar. Il s'agit d'un lycée majoritairement général
(avec toutefois quelques BTS). Notre intervention avait lieu pendant le temps d'AP (Accompagnement Personnalisé), ce
temps permet aux élèves de revenir sur les notions mal comprises ou d'envisager un approfondissement sur un sujet qui
les intéresse.

Au cours de ce mémoire, je présenterais tout d'abord l'organisation générale de l'équipe pédagogique puis les trois
temps de notre intervention. Enfin, je reviendrais sur mon ressenti avant de conclure.
