\documentclass[a4paper,11pt]{article}

	\usepackage[utf8]{inputenc}
	\usepackage[T1]{fontenc}
	\usepackage[french]{babel}

	\usepackage[top=1.5cm,right=1.5cm,left=1.5cm,bottom=1.5cm]{geometry}

	\title{Aux Etats-Unis, un projet de puce scolaire géolocalisable provoque une controverse judiciaire}
	\author{}
	\date{}

\begin{document}
\maketitle

Un contentieux judiciaire retarde l'ordre de renvoi d'une élève Texane pour avoir refusé de porter une puce radio
qui tracerait ses déplacements.

Des raisons religieuses ont poussé Andrea Hernandez à arrêter de porter la puce qui révèlait où elle se trouvait sur
le campus de son école.

Les puces ont été introduites pour localiser les étudiants et aider à raffermir le contrôle des subventions de l'école.

Une cour texane a accédé à une injonction déposée par un groupe de défense des droits civils dans l'attente d'une
audition à propos de l'utilisation des puces.

Les badges d'identification contenant les puces radio ont commencés à être introduits au début de l'année scolaire 2012
dans les écoles gérées par le NISD (Northside Independent School District) de San Antonio. Les puces traçantes donnent au
NISD une meilleure idée du nombre d'étudiants assistant aux cours chaque jour -- la moyenne quotidienne des présents
dictant le montant reçu par le NISD des caisses de l'état.

\section*{La marque du diable}

L'introduction des puces a entraîné des manifestations de la part d'étudiants du lycée Jon Jay High School -- une des
deux écoles pilotes pour les puces parmi les 112 dans le secteur\footnote{je ne vois pas de bonne traduction
pour \textit{catchment area}} du NISD.

Mlle Hernandez a refusé de porter le badge en ce qu'il est contraire à ses convictions religieuses, d'après les dossiers
du tribunal.
Porter un tel code-barre peut être vu comme une marque du diable\footnote{je ne pense pas que \textit{"mark of the
beast"} soit traduisible littéralement} comme elle est décrite dans la Bible (Revelation, 13) a déclaré le père de Mlle
Hernandez au magazine Wired dans une interview.

Le NISD a suspendu Mlle Hernandez et a annoncé qu'elle ne pourrait pas retourner à la John Jey High School à moins
qu'elle ne porte le badge d'identification contenant la puce radio. D'autre part, il a été précisé que Mlle Hernandez
pouvait fréquenter d'autres écoles dans le quartier\footnote{est-ce une bonne traduction de \textit{district} ?} n'ayant
pas encore rejoint le project d'identification radio.

L'institut Rutherford, un collectif se battant pour les libertés, a rejoint les manifestations et s'est déplacé jusqu'au
tribunal pour réclamer une ordonnance mettant fin à la suspension de Mlle Hernandez à l'attention du NISD.

Un juge du tribunal local a accédé à la requête d'injonction, Mlle Hernandez peut ainsi retourner à l'école, il a aussi
demandé une audience la semaine prochaine à propos du projet de puce radio.

L'institut Rutherford affirme que la suspension prononcé par le NISD viole les lois Texanes sur la liberté de culte et
les amendements de la constitution des Etats-Unis à propos de la liberté d'expression.

"La volonté du tribunal d'accorder une ordonnnance temporaire est un premier pas prometteur, mais il reste une longue
route à faire -- pas seulement dans ce cas, mais à propos de l'état d'esprit en général qui trouve normal que tout le
monde soit être surveillé et contrôlé." avance John Whitehead, président de l'institut Rutherford dans une déclaration.

M. Whitehead affirme que l'étiquettage des étudiants et leur localisation est une première étape vers la création d'une
"citoyenneté docile".

"Ces programmes de 'localisateurs à étudiants' ont pour but d'habituer les élèves à vivre dans un état de surveillance
totale où il n'y aurait pas de vie privée, et où, où que vous alliez et quoi que vous envoyiez par sms ou email, tout
serait vérifié par le gouvernement" a-t-il ajouté.

\end{document}
